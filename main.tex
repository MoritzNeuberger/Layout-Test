\documentclass[encoding=utf8,british]{tumphthesis}
% \documentclass[pstricks,siunitx,addfonts,theorem,font=palatino,british]{tumphthesis}

% Das folgende Paket dient lediglich dazu, den Blindtext "Lorem ipsum ..."
% auszugeben und kann in einer echten Abschlussarbeit natürlich weggelassen 
% (oder auskommentiert) werden.
\usepackage{lipsum}

% Die Metadaten der Abschlussarbeit werden auf dem Deckblatt gedruckt und
% in dem PDF eingetragen.
\subject{Abschlussarbeit im Bachelorstudiengang Physik}
\title{Titel der Abschlussarbeit}
%\subtitle{\foreignlanguage{british}{Title in English}}
\author{Sheldon Cooper}
\date{31.~Juli 2011}
%\cooperators{Max-Planck-Institut für Physik}

% Auf der Rückseite des Deckblatts können Themensteller, Zweitgutachter 
% und Tag der mündlichen Prüfung vermerkt werden.
\lowertitleback{Erstgutachter (Themensteller): Prof.\ A.~Kabelschacht\\
Zweitgutachter: Prof.\ S.~Preuss}

\begin{document}
% Ist die Arbeit auf Englisch verfasst, hier die Sprache umschalten. 
% Die Sprache muss als Klassenoption angegeben sein.
\selectlanguage{british}

\frontmatter
\maketitle
\tableofcontents

\chapter{Vorwort}
\lipsum[1]
\todo{noch ausarbeiten!}

\mainmatter
\chapter{Titel des ersten Kapitels}
\section{Erster Abschnitt}
\lipsum[2-5]\cite{schwabl-qqi2002}

Und noch etwas \emph{betontes}.

\section{Zweiter Abschnitt}
\lipsum[6] Test\cite{Setare:2013dra}
\begin{figure}
	\centering
	\includegraphics[width=\textwidth]{tumlogo}
	\caption{\label{fig:test}Test}
\end{figure}
\subsection{Unterabschnitt}
\lipsum[7]\cite{schwabl-qqi2002,schwabl-qffi2002}
\subsubsection{Unterunterabschnitt}
\paragraph{Absatz} \lipsum[8]

\appendix
\chapter{Erstes Kapitel im Anhang}
\lipsum[9-20]

\backmatter
\printbibliography


\end{document}